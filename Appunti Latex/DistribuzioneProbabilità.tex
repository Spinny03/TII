\documentclass[12pt]{article}
\usepackage[italian]{babel}
\usepackage{geometry}
\usepackage{amsmath}
\usepackage{graphicx}
\usepackage{amssymb}
\usepackage{pgfplots}
\usepackage[dvipsnames]{xcolor}
\usepgfplotslibrary{fillbetween}

\pgfplotsset{compat=1.17, legend pos=south east}

\geometry{margin=2cm}

\title{Distribuzione di probabilità}
\author{Lorenzo Vaccarecci}
\date{22 Marzo 2024}

\graphicspath{{./Immagini/}}

\begin{document}
\maketitle
\section{Distribuzione di una funzione di variabile casuale}
\textbf{Esempio}
\begin{quote}
Abbiamo $X \text{ distribuita uniformementre tra 0 e 1,abbiamo } f(x)=1 \text{ e } F(x)=x$. Se $Y=X^{n}:$\\
\(
  F_{Y}(y)=P(Y\leq y)=P(X?{n}\leq y)=P(X\leq y^{\frac{1}{n}})=F(y^{\frac{1}{n}})=y^{\frac{1}{n}}  
\)
\end{quote}
Dall'esempio possiamo ricavare le formule generali per $Y(g(x))$:
\begin{equation*}
    F_{Y}(y)=F(g^{-1}(y))
\end{equation*}
\begin{equation*}
    f_{Y}(y)=f(y)\cdot \frac{d}{dy}(g^{-1}(y))^{*}
\end{equation*}
\textit{*Derivata della funzione composta.}
\section{Distribuzione normale (o Gaussiana)}
$X=\mathcal{N}(\mu,\sigma^{2})\quad\mu \text{ e } \sigma > 0$
\begin{equation*}
    f(x)=\frac{1}{\sigma \sqrt{2\pi}}e^{-\frac{(x-\mu)^2}{2\sigma^{2}}}
\end{equation*}
\subsection{Valore atteso e varianza}
Il valore atteso (di una normale standard $Z=\mathcal{N}(0,1)$) è: 
\begin{equation*}
    \mathbb{E}[Z]=\int zf(x)dz=\int z\frac{1}{\sigma \sqrt{2\pi}}e^{-\frac{(x-\mu)^2}{2\sigma^{2}}}=\int \textcolor{OliveGreen}{z}\frac{\textcolor{OliveGreen}{e^{-\frac{z^{2}}{2}}}}{\sqrt{2\pi}}dz=0
\end{equation*}
L'area in verde è nulla, quindi il valore atteso è 0.
\begin{center}
    \begin{tikzpicture}
        \begin{axis}[axis lines=center,xlabel=$x$,ylabel=$\mathcal{N}$]
        \addplot[domain=-10:10,samples=100,color=blue]{exp(-x^2/2)};
        \addplot[domain=-5:5,samples=100,color=red, name path=B]{x};
        \addplot[domain=-10:10,samples=100,color=violet, name path=A]{x*exp(-x^2/2)};
        \addlegendentry{$e^{-\frac{z^{2}}{2}}$}
        \addlegendentry{$z$}
        \addlegendentry{$z\cdot e^{-\frac{z^{2}}{2}}$}

        \path[name path=axis] (axis cs:0,0) -- (axis cs:1,0);
        \addplot[OliveGreen] fill between[of=A and axis, soft clip={domain=-5:5}];
        \end{axis}
    \end{tikzpicture}
\end{center}
La varianza è (per parti):
\begin{equation*}
    Var(Z)=\int \frac{z^2e^{\frac{-z^{2}}{2}}}{\sqrt{2\pi}}dz=\dots=1
\end{equation*}
\end{document}
\documentclass[12pt]{article}
\usepackage[italian]{babel}
\usepackage{geometry}
\usepackage{amsmath}
\usepackage{graphicx}
\usepackage{amssymb}

\geometry{margin=2cm}

\title{Introduzione al calcolo combinatorio}
\author{Lorenzo Vaccarecci}
\date{29 Febbraio 2024}

\graphicspath{{./Immagini/}}
\newtheorem{example}{Esempio}

\begin{document}
\maketitle
\section{Esperimento}
Azione con \(N\) risultati possibili.\\
\begin{example}
Lancio di un dado. \(N=6\)\\
Lancio di una moneta.\(N=2\)\\
\end{example}
\section{Principio Base}
2 esperimenti con \(N\) e \(M\) risultati possibili, i risultati possibili sono \(N\cdot M\)
\begin{description}
    \item[Esercizio] Quante combinazioni sono possibili con una targa? Quante senza duplicati?
    \item[] Nella targa ci sono 4 lettere (nell'alfabeto 26) e 3 cifre (10 numeri da 0 a 9).
    \item[] Le combinazioni possibili sono \(26^4\cdot 10^3\)
    \item[] Senza duplicati: \(26\cdot 25\cdot 24\cdot 23\cdot 10\cdot 9\cdot 8\)
\end{description}
\section{Permutazione}
Ordinamento di \(N\) oggetti.\\
\(N!=N\cdot(N-1)\cdot(N-2)\dots1\)
\begin{description}
    \item[Esercizio] Numero ordinamenti su uno scaffale e raggruppati di 2 libri di chimica, 3 di fisica, 4 di matematica e 5 di informatica.
    \item[] Numero di ordinamenti \(=14!\)
    \item[] Raggruppati \(2!3!4!5! \cdot 4!\) (4 è il numero di gruppi)
\end{description}
\section{Disposizione}
Ordinamento di \(i\) oggetti tra \(N\) oggetti.\\
\(N\cdot(N-1)\dots(N-i+1)=\frac{N!}{(N-i)!}\)
\begin{description}
    \item[Esercizio] Anagrammi
    \item[Cinema] \(6!\)
    \item[Errore] \(\frac{6!}{3!2!}\) (a denominatore abbiamo le lettere duplicate 3 "R" e 2 "E", a numeratore il numero totale di lettere)
\end{description}
\section{Combinazione}
Scelta di \(i\) oggetti tra \(N\) oggetti.\\
\(\frac{N!}{i!(N-i)!}=\binom{N}{i}\) detto anche "Binomio di Newton"
\begin{description}
    \item[Esercizio] Comitato di 3 persone su 20.
    \item[] \(\frac{20!}{3!17!}=\frac{20\cdot 19 \cdot 18}{3\cdot 2}= 60\cdot 19\) 
\end{description}
\end{document}
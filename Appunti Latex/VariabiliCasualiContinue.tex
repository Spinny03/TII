\documentclass[12pt]{article}
\usepackage[italian]{babel}
\usepackage{geometry}
\usepackage{amsmath}
\usepackage{graphicx}
\usepackage{amssymb}
\usepackage{pgfplots}

\geometry{margin=2cm}

\title{Variabili Casuali Continue}
\author{Lorenzo Vaccarecci}
\date{21 Marzo 2024}

\graphicspath{{./Immagini/}}

\begin{document}
\maketitle
\section{Funzione densità di probabilità}
Una variabile casuale $X$ è continua se esiste una funzione $f:\mathbb{R}\rightarrow\mathbb{R}^{+}$ tale che
\begin{equation*}
    P(X\in B)=\int_{B}f(x)dx
\end{equation*}
e.g.
\begin{equation*}
    B=[-\epsilon,\epsilon] \quad P(X\in B)=\int_{-\epsilon}^{\epsilon}f(x)dx
\end{equation*}
su ogni sottoinsieme misurabile $B\subset \mathbb{R}$. La funzione $f$ è la \textit{densità di probabilità, o pdf}. La \textit{pdf} è parente stretta della \textit{pmf}
\\Se prendiamo tutto:
\begin{equation*}
    \int f(x)dx = 1
\end{equation*}
\begin{equation*}
    \mathbb{E}[X]=\int xf(x)dx
\end{equation*}
\begin{equation*}
    \mathbb{E}[g(x)]=\int g(x)f(x)dx
\end{equation*}
\begin{equation*}
    Var(X)=\int (x-\mathbb{E}[X])^{2}f(x)
\end{equation*}
* La $x$ equivale alla $i$ delle variabili discrete e la $f(x)$ equivale a $p_{i}$, infatti le formule sono molto simili solo che al posto della sommatoria c'è l'integrale.
\section{Funzione di distribuzione cumulata}
La \textit{funzione di distribuzione cumulata} $F:\mathbb{R}\rightarrow[0,1]$, o cdf, è definita $\forall a \in \mathbb{R}$ come
\begin{equation*}
    F(x)=\int_{-\infty}^{a}f(t)dt
\end{equation*}
Per il teorema fondamentale del calcolo integrale:
\begin{equation*}
    \frac{d}{dx}(F(x))=f(x)
\end{equation*}
\end{document}